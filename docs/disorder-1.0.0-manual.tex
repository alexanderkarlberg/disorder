% =========================================================================
% SciPost LaTeX template
% Version 2021-08
%
% Submissions to SciPost Journals should make use of this template.
%
% INSTRUCTIONS: simply look for the `TODO:' tokens and adapt your file.
%
% You can also make use of our empty "skeleton" templates for each Journal,
% e.g. SciPostPhys_skeleton.tex
% =========================================================================
\documentclass[submission, PhysCodeb]{SciPost}

% Prevent all line breaks in inline equations.
\binoppenalty=10000
\relpenalty=10000

\hypersetup{
    colorlinks,
    linkcolor={red!50!black},
    citecolor={blue!50!black},
    urlcolor={blue!80!black}
}

\usepackage[bitstream-charter]{mathdesign}
\urlstyle{sf}

% Fix \cal and \mathcal characters look (so it's not the same as \mathscr)
\DeclareSymbolFont{usualmathcal}{OMS}{cmsy}{m}{n}
\DeclareSymbolFontAlphabet{\mathcal}{usualmathcal}

\newcommand{\hoppet}{{\tt HOPPET}}
\newcommand{\disent}{{\tt disent}}
\newcommand{\provbfh}{{\tt proVBFH}}
\newcommand{\disaster}{{\tt disaster++}}
\newcommand{\nlojet}{{\tt nlojet++}}
\newcommand{\as}{\alpha_{\mathrm{s}}}

\begin{document}
\begin{flushright}
CERN-TH-2023-???
\end{flushright}
\begin{center}{\Large \textbf{
disorder: Deep inelastic scattering at high orders\\
}}\end{center}

\begin{center}
Alexander Karlberg
\end{center}

\begin{center}
CERN, Theoretical Physics Department, CH-1211 Geneva 23, Switzerland
\\
{\small \sf alexander.karlberg@cern.ch}
\end{center}

\begin{center}
\today
\end{center}

% For convenience during refereeing (optional),
% you can turn on line numbers by uncommenting the next line:
%\linenumbers
% You should run LaTeX twice in order for the line numbers to appear.

\section*{Abstract}
{\bf
% TODO: write your abstract here.
We present a Fortran 77/95 code capable of computing QCD corrections
in deep inelastic scattering (DIS). The code uses the
projection-to-Born method to combine an exclusive DIS + 1 jet
calculation with the inclusive DIS structure functions, thereby
obtaining fully differential DIS predictions at
$\mathcal{O}(\as^2)$. The code is lightweight and fast, and yet
includes the most common functionalities found in typical perturbative
QCD programs, like automatic renormalisation and factorisation scale
uncertainties, options to run and combine multiple seeds, and
interfaces to {\tt fastjet} and {\tt LHAPDF}. Due to the underlying
exclusive DIS + 1 jet code, the program also provides stable results
in the infrared, relevant for extracting logarithmic coefficients for
analytic resummations. As a by-product the code provides access
to the DIS structure functions and (reduced) cross sections up to
$\mathcal{O}(\as^3)$.}


% TODO: include a table of contents (optional)
% Guideline: if your paper is longer that 6 pages, include a TOC
% To remove the TOC, simply cut the following block
\vspace{10pt}
\noindent\rule{\textwidth}{1pt}
\tableofcontents\thispagestyle{fancy}
\noindent\rule{\textwidth}{1pt}
\vspace{10pt}

\section{Introduction}
\label{sec:intro}
Deep inelastic scattering (DIS) is one of the best understood
processes in perturbative Quantum Chromo Dynamics (QCD). It is one of
few processes for which there exists an exact factorisation
theorem~\cite{Collins:1987pm,Collins:1989gx}, and the massless
unpolarised hard perturbative coefficients have been computed through
an impressive three
loops~\cite{SanchezGuillen:1990iq,vanNeerven:1991nn,Zijlstra:1992qd,Zijlstra:1992kj,vanNeerven:1999ca,vanNeerven:2000uj,Moch:1999eb,Moch:2004xu,Vermaseren:2005qc,Vogt:2006bt,Moch:2007rq,Davies:2016ruz}
with progress on the four loop results presented recently in
Ref.~\cite{Moch:2022frw}. Together with the three-loop results for the
splitting functions~\cite{Moch:2004pa,Vogt:2004mw} and the four-loop
$\beta$-function~\cite{vanRitbergen:1997va,Czakon:2004bu} for the
running QCD coupling this allows for the determination of the proton
structure functions at the next-to-next-to-next-to-leading order
(N3LO)\footnote{Technically the four-loop splitting functions are
needed to claim this accuracy. Recent progress in determining those
can be found in
Refs.\cite{Moch:2021qrk,Falcioni:2023luc,Falcioni:2023vqq,Gehrmann:2023cqm}.}
These can be combined with an exclusive next-to-next-to-leading order
(NNLO) DIS+1 jet computation to obtain fully differential N3LO
predictions in DIS, as was done by the NNLOJET
collaboration~\cite{Currie:2018fgr,Gehrmann:2018odt}. This compuation
uses the projection-to-Born method, which was first introduced in the
context of NNLO Vector Boson Fusion (VBF)
production~\cite{Cacciari:2015jma}.



Some references: \hoppet{}~\cite{Salam:2008qg}, NNLO coefficient
functions~\cite{vanNeerven:1999ca,vanNeerven:2000uj}, N3LO coefficient
functions~\cite{Moch:2004xu,Vermaseren:2005qc,Moch:2008fj,Davies:2016ruz},
\provbfh{}~\cite{Cacciari:2015jma}, \disent{}~\cite{Catani:1996vz}
with bug fix fromRefs.~\cite{Borsa:2020ulb,Borsa:2020yxh},
\nlojet{}~\cite{Nagy:2001xb}, \disaster{}~\cite{Graudenz:1997gv}

\section{The DIS process}
\label{sed:dis}

\section{Running the program}
\label{sec:running}

\section{Benchmarks and results}
\label{sec:benchmarks}

\section{Conclusion}
\label{sec:conclusion}

\section*{Acknowledgements}
The author is grateful for discussions with Gavin Salam regarding {\tt disent}. 


\begin{appendix}

  \section{An appendix}

\end{appendix}

\bibliography{bibliography.bib}

\nolinenumbers

\end{document}
