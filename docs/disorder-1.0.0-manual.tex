% =========================================================================
% SciPost LaTeX template
% Version 2021-08
%
% Submissions to SciPost Journals should make use of this template.
%
% INSTRUCTIONS: simply look for the `TODO:' tokens and adapt your file.
%
% You can also make use of our empty "skeleton" templates for each Journal,
% e.g. SciPostPhys_skeleton.tex
% =========================================================================
\documentclass[submission, PhysCodeb]{SciPost}
% Prevent all line breaks in inline equations.
\binoppenalty=10000
\relpenalty=10000

\hypersetup{
    colorlinks,
    linkcolor={red!50!black},
    citecolor={blue!50!black},
    urlcolor={blue!80!black}
}

\usepackage[bitstream-charter]{mathdesign}
\usepackage{listings}
% Define special colors
\definecolor{comment}{rgb}{0,0.3,0}
\definecolor{identifier}{rgb}{0.0,0,0.3}

\lstset{language=bash}
\lstset{
  columns=flexible,
  basicstyle=\tt\footnotesize,
  keywordstyle=,
  identifierstyle=\color{black},
  commentstyle=\tt\color{comment},
  mathescape=true,
  texcl=true,
  escapebegin=\color{comment},
  showstringspaces=false,
  keepspaces=true
}
\urlstyle{sf}

% Fix \cal and \mathcal characters look (so it's not the same as \mathscr)
\DeclareSymbolFont{usualmathcal}{OMS}{cmsy}{m}{n}
\DeclareSymbolFontAlphabet{\mathcal}{usualmathcal}

\newcommand{\hoppet}{{\sc hoppet}}
\newcommand{\disent}{{\tt disent}}
\newcommand{\disorder}{{\tt disorder}}
\newcommand{\provbfh}{{\tt proVBFH}}
\newcommand{\disaster}{{\tt disaster++}}
\newcommand{\nlojet}{{\tt nlojet++}}
\newcommand{\as}{\alpha_{\mathrm{s}}}
\newcommand{\dd}{\mathrm{d}}
\newcommand{\NC}{\mathrm{NC}}
\newcommand{\CC}{\mathrm{CC}}

\begin{document}
\begin{flushright}
CERN-TH-2023-???
\end{flushright}
\begin{center}{\Large \textbf{
disorder: Deep inelastic scattering at high orders\\
}}\end{center}

\begin{center}
Alexander Karlberg
\end{center}

\begin{center}
CERN, Theoretical Physics Department, CH-1211 Geneva 23, Switzerland
\\
{\small \sf alexander.karlberg@cern.ch}
\end{center}

\begin{center}
\today
\end{center}

% For convenience during refereeing (optional),
% you can turn on line numbers by uncommenting the next line:
%\linenumbers
% You should run LaTeX twice in order for the line numbers to appear.

\section*{Abstract}
{\bf
% TODO: write your abstract here.
We present a Fortran 77/95 code capable of computing QCD corrections
in deep inelastic scattering (DIS). The code uses the
Projection-to-Born method to combine an exclusive DIS + 1 jet
calculation with the inclusive DIS structure functions, thereby
obtaining fully differential DIS predictions at
$\mathcal{O}(\as^2)$. The code is lightweight and fast, and yet
includes the most common functionalities found in typical perturbative
QCD programs, like automatic renormalisation and factorisation scale
uncertainties, options to run and combine multiple seeds, and
interfaces to {\tt fastjet} and {\tt LHAPDF}. Due to the underlying
exclusive DIS + 1 jet code, the program also provides stable results
in the infrared, relevant for extracting logarithmic coefficients for
analytic resummations. As a by-product the code provides access
to the DIS structure functions and (reduced) cross sections up to
$\mathcal{O}(\as^3)$.}


% TODO: include a table of contents (optional)
% Guideline: if your paper is longer that 6 pages, include a TOC
% To remove the TOC, simply cut the following block
\vspace{10pt}
\noindent\rule{\textwidth}{1pt}
\tableofcontents\thispagestyle{fancy}
\noindent\rule{\textwidth}{1pt}
\vspace{10pt}
\newpage
\section{Introduction}
\label{sec:intro}
Deep inelastic scattering (DIS) is arguably one of the best understood
processes in perturbative QCD. It is one of few processes for which
there exists an exact factorisation
theorem~\cite{Collins:1987pm,Collins:1989gx}, and the massless
unpolarised hard perturbative coefficients have been computed through
an impressive three
loops~\cite{SanchezGuillen:1990iq,vanNeerven:1991nn,Zijlstra:1992qd,Zijlstra:1992kj,vanNeerven:1999ca,vanNeerven:2000uj,Moch:1999eb,Moch:2004xu,Vermaseren:2005qc,Vogt:2006bt,Moch:2007rq,Davies:2016ruz}
with progress on the four loop results presented recently in
Ref.~\cite{Moch:2022frw}. Together with the three-loop results for the
splitting functions~\cite{Moch:2004pa,Vogt:2004mw} and the four-loop
$\beta$-function~\cite{vanRitbergen:1997va,Czakon:2004bu} for the
running coupling this allows for the determination of the proton
structure functions at the next-to-next-to-next-to-leading order
(N3LO).\footnote{Technically the four-loop splitting functions are
needed to claim this accuracy. Recent progress in determining those
can be found in
Refs.\cite{Moch:2021qrk,Falcioni:2023luc,Falcioni:2023vqq,Gehrmann:2023cqm}.}
The structure functions can be combined with an exclusive
next-to-next-to-leading order (NNLO) DIS + 1 jet computation to obtain
fully differential N3LO predictions in DIS, as was done by the NNLOJET
collaboration~\cite{Currie:2018fgr,Gehrmann:2018odt}. This computation
uses the Projection-to-Born (P2B) method, which was first introduced
in the context of NNLO Vector Boson Fusion (VBF)
production~\cite{Cacciari:2015jma}.

Despite this impressive theoretical progress, there are few publically
available computer codes from which one can obtain
fast and reliable high-order cross section predictions. This paper
and associated Fortran code seeks to address that.

One of the advantages of the P2B method is that it is rather agnostic
towards the details of the underlying exclusive computation, and it is
hence possible to ``upgrade'' existing fixed-order codes to one order
higher. Historically the most succesful fixed-order codes have been
\disent{}~\cite{Catani:1996vz}, \disaster{}~\cite{Graudenz:1997gv},
and \nlojet{}~\cite{Nagy:2001xb} which are all next-to-leading order
(NLO) accurate. Discrepancies between \disent{} and fixed-order
coefficients from analytical resummation were initially observed in
Refs.\cite{Antonelli:1999kx,Dasgupta:2002dc}, and were only recently
understood to be due to a bug in one of the dipole terms for the gluon
channel~\cite{Borsa:2020ulb,Borsa:2020yxh}.

With the bug fixed we can use \disent{} as the underlying exclusive
NLO code together with the NNLO structure functions
from~\hoppet{}~\cite{Salam:2008qg,BertoneKarlberg}.\footnote{The
combination of \disent{} and \hoppet{} naturally leads to a Fortran
code. For a C++ alternative, based on publically available codes, one
could have started from either \disaster{} or \nlojet{} and used the
structure functions as implemented in
APFEL++~\cite{Bertone:2013vaa,Bertone:2017gds}.} Advantages of using
\disent{} are its well-known efficiency and numerical stability. The
resulting program is dubbed \disorder{} keeping in the spirit of
previous names for fixed-order DIS codes.\footnote{{\tt
  dispatch}~\cite{Dasgupta:2002dc} deserves an honorable mentioning in
this context.} Since the program relies on \hoppet{} for the structure
functions, the inclusive DIS cross section can actually be obtained
one order higher, since the structure functions were implemented at
N3LO already in the context of the {\tt proVBFH}
code~\cite{Cacciari:2015jma,Dreyer:2016oyx,Dreyer:2018qbw,Dreyer:2018rfu}. The
\disorder{} program therefore has two more or less separate use cases:
1. The computation of fully exclusive NNLO photon-mediated DIS 2. The
computation of inclusive/reduced neutral- (NC) and charged-current
(CC) DIS at N3LO accuracy.

The program is designed to be user-friendly with a simple command line
interface. The program prints cross sections and all important
run-parameters to the screen and disk, allowing a user to acquire
cross sections with very little effort. It comes with an interface to
{\tt fastjet} and {\tt LHAPDF} and uses the histogramming package from
the {\tt POWHEG-BOX}~\cite{Alioli:2010xd} for easy analysis. The code
can also compute renormalisation and factorisation scale uncertainties
on-the-fly. For inclusive cross sections the code provides results in
a matter of seconds\footnote{The majority of this time is taken up by
the structure function initialisation inside \hoppet{} rather than the
integration of the cross section.} even at N3LO with uncertainties
typically below the permille level. For exclusive quantities the code
can be run on a laptop at NLO and depending on the analysis, and the
laptop, even at NNLO.

It should be pointed out that the code does not provide any
theoretical advances on its own. As described above, the structure
functions themselves have been known for a little while, and the P2B
method has already been applied to DIS, even one order higher than
here. It is however the author's opinion, that public code is
extremely valuable for both the experimental and theoretical
communities, and that providing documentation in the form of this
article will enable the wide use of the code.

The paper is structured as follows: In section~\ref{sec:dis} we review
the DIS process and kinematics as implemented in \disorder{}. In
section~\ref{sec:running} we provide details on how to run \disorder{}
and in section~\ref{sec:results} we show a few results from the
program. Finally we conclude in section~\ref{sec:conclusion}.



\section{Basics of the DIS process}
\label{sec:dis}
In this section we first give some standard definitions for kinematics
and cross sections in DIS. Along the way we specify the conventions
that are used in the \disorder{} code, and provide some details on the
P2B method as applied here.

At leading order (LO) the DIS process is
the scattering of a massless (anti-)quark $q$ off a massless
\mbox{(anti-)lepton} $l$ via the exchange of a photon or electroweak
gauge boson $V$ of virtuality $Q^2$. Denoting the external
four-momenta by $k_i$ (incoming lepton), $k_f$ (outgoing lepton),
$p_i$ (incoming quark), and $p_f$ (outgoing quark) we can define the
lorentz invariant DIS variables $x$, $Q^2$, and $y$, given by
\begin{align}
  Q^2 = -q^2 = -(k_i - k_f)^2, \qquad  x = \frac{Q^2}{2 P \cdot q},
  \qquad  y = \frac{P \cdot q}{P \cdot k_i} = \frac{p_i \cdot q}{p_i \cdot k_i} ,
  \label{eq:dis-variables}
\end{align}
where $P$ is the proton four-momentum. As can be seen these kinematics
are fully specified by the proton and lepton momenta. This is also
true beyond LO. LO. DIS is most often analysed in the Breit-frame
which is specified\footnote{For the explicit transformation between
lab- and Breit-frames we follow Appendix 7.11 in
Ref.~\cite{Devenish:2004pb}.}  by requiring that $2 x \vec{P} +
\vec{q} = 0$. In this frame the mediated vector boson has zero energy
component and is anti-aligned with the incoming parton. Explicitly in
\disorder{} the Breit-frame at LO is
\begin{align}
  \label{eq:BreitLO}
k_i &= \frac{Q}{2}\left(\frac{2-y}{y},\frac{2\sqrt{1-y}}{y},0,-1 \right), \quad  p_i = \frac{Q}{2}\left(1,0,0,+1 \right) \\ \notag
k_f &= \frac{Q}{2}\left(\frac{2-y}{y},\frac{2\sqrt{1-y}}{y},0,+1 \right), \quad  p_f = \frac{Q}{2}\left(1,0,0,-1 \right) 
\end{align}
where four-momenta are given as $(E,p_x,p_y,p_z)$. The resulting
vector $q$ is hence given by $Q(0,0,0,-1)$. In the lab frame we align
the parton with the positive $z$-axis and hence the lepton with the
negative
\begin{align}
\tilde{k}_i &= E_\mathrm{l}\left(1,0,0,-1 \right), \quad  \tilde{p}_i = x E_\mathrm{h}\left(1,0,0,+1 \right) 
\end{align}
where $E_\mathrm{h}$ is the energy of the incoming hadron. Using the
definitions of eq.~\eqref{eq:dis-variables} one finds that in the lab
frame
\begin{equation}
  \tilde{q} = \left(y(E_\mathrm{l} - x E_\mathrm{h}), -Q\sqrt{1-y},0, -y(E_\mathrm{l} - x E_\mathrm{h})\right) 
\end{equation}
and the outgoing momenta then simply follow from momentum conservation
\begin{equation}
  \label{eq:LabLO}
  \tilde{k}_f = \tilde{k}_i - \tilde{q}, \quad   \tilde{p}_f = \tilde{p}_i + \tilde{q}\,. 
\end{equation}

The inclusive cross section for DIS can be split into a NC
contribution, from $e^\pm p \to e^\pm + X $ scattering, and a CC
contribution from $e^\pm p \to \nu + X$\footnote{Here
$\nu=\{\nu_e,\bar{\nu}_e\}$ for $e^-$ and $e^+$ respectively. One can
of course also consider incoming neutrinos which does not change the
discussion here.} scattering. The unpolarised NC cross section can be
written as
\begin{equation}
\frac{\dd\sigma_{\NC}^\pm}{\dd x \dd Q^2} =   \frac{2\pi\alpha^2}{xQ^4} \left[y_+ F_2^{\NC} \mp y_- x F_3^\NC - y^2 F_L^\NC\right],\,
\label{eq:NCsigma}
\end{equation}
where $y_\pm=1\pm(1-y)^2$, $\alpha$ is the fine structure constant and
$F_i^\NC$ can be expressed in terms of the usual proton structure
functions
\begin{align}
  F_{i}^\NC &= F_{i}^\gamma  - v_e \Gamma_{\gamma Z} F_{i}^{\gamma Z} + (v_e^2 + a_e^2)\Gamma_{Z} F_{i}^{Z},\, \quad i =2,L \\ \notag
  F_{3}^\NC &= - a_e \Gamma_{\gamma Z} F_{3}^{\gamma Z} + 2v_e a_e\Gamma_{Z} F_{3}^{Z}\,.
\end{align}
Here $v_e = -\frac12 + 2 \sin^2\theta_W$ is the vector and
$a_e=\frac12$ is the axial-vector coupling, $M_Z$ is the Z boson mass,
$\Gamma_{\gamma Z} = \frac{Q^2}{\sin^2 2\theta_W(Q^2+M_Z^2)}$, and
$\Gamma_Z=\Gamma_{\gamma Z}^2$. $\theta_W$ is the weak mixing
angle. In \disorder{} the electroweak parameters are fixed by
$\alpha$, $M_W$, and $M_Z$ through the tree-level relations
\begin{align}
  \sin^2\theta_W = 1-\frac{M_W^2}{M_Z^2}, \qquad G_F = \frac{\pi\alpha}{\sqrt{2}m_W^2\sin^2\theta_W},
\end{align}
with $G_F$ the Fermi constant.

Similarly we define the unpolarised CC cross section as
\begin{equation}
\frac{\dd\sigma_{\CC}^\pm}{\dd x \dd Q^2} =   \frac{\pi\alpha^2}{8\sin^4\theta_W x}\left[\frac{1}{M_W^2 + Q^2}\right]^2 \left[y_+ F_2^{\CC} \mp y_- x F_3^\CC - y^2 F_L^\CC\right],\,
\label{eq:CCsigma}
\end{equation}
where $M_W$ is the mass of $W$ boson and the CC structure functions
are now simply given by the $W$ ones
\begin{equation}
  F_2^\CC = F_2^{W^\pm}, \quad   F_L^\CC = F_L^{W^\pm}, \quad   F_3^\CC = F_3^{W^\pm}\,.
\end{equation}
The exact definitions of all proton structure functions inside
\hoppet{} up to N3LO can be found in
Refs.~\cite{Salam:2008qg,BertoneKarlberg}. In DIS it is also customary
to define the dimensionless \emph{reduced} NC and CC cross sections
by~\cite{H1:2012qti}
\begin{align}
  \label{eq:reducedsigma}
  \tilde{\sigma}_\NC^\pm(x,Q^2) &= \frac{xQ^4}{2\pi\alpha^2}\frac{1}{y_+}\frac{\dd\sigma_{\NC}^\pm}{\dd x \dd Q^2} \\ \notag
  \tilde{\sigma}_\CC^\pm(x,Q^2) &= \frac{8\sin^4\theta_W x}{\pi\alpha^2}\left[M_W^2+Q^2\right]^2\frac{\dd\sigma_{\CC}^\pm}{\dd x \dd Q^2}\,.
\end{align}
\disorder{} provides direct access to all the cross sections in
eqs.~\eqref{eq:NCsigma},~\eqref{eq:CCsigma},
and~\eqref{eq:reducedsigma} at N3LO accuracy. In principle one can
also access the NC and CC structure functions, although they are
currently only computed as an intermediate step to construct the cross
sections.

\subsection{Applying P2B}
\label{sec:P2B}
The structure functions are by definition inclusive in all radiation
and can therefore only provide predictions for quantities, like the
inclusive cross sections, which depend on the Born kinematics of
eq.~\eqref{eq:dis-variables} only. If we instead evaluate the
structure functions on an observable sensitive to emissions, e.g. the
transverse momentum of the hardest jet in the lab frame, we see that
this will not give the right answer, as the real emissions are not
included with their correct kinematics. 


The P2B method lifts this restriction by effectively replacing the
Born-kinematics real-emission contributions in the structure functions
with the correct kinematics ones. In practice whenever \disent{}
returns an event with some weight, we bin it once according to the
true kinematics, and again projecting the kinematics to the underlying
Born changing the sign of the weight. This last term, upon integration,
will exactly cancel the real contribution in the structure functions,
whereas the first term will provide the correct real matrix element.

The projections themselves are trivial because, as outlined above, the
Born kinematics are fully specified at all orders by the lepton (and
proton) momenta, as given in
eqs.~\eqref{eq:BreitLO}--\eqref{eq:LabLO}. Hence computing the
projections adds very little computational effort to the cross section
calculation.

At this point it is worth reminding the reader that \disent{} only
includes the photon-mediated NC, and hence \disorder{} only provides
exclusive predictions for this channel. It should in principle be
possible to extend the code to include Z-mediation (including
interferences) and CC, but we leave this for future work.


\section{Running \disorder{}}
\label{sec:running}
In this section we give instructions on how to compile and run
\disorder{}, giving a few examples of the use of the most common
command line arguments.

\subsection{Compiling and prerequisites}
To compile \disorder{} both \hoppet{} and
LHAPDF~\cite{Buckley:2014ana} have to be installed on the machine. If
both are installed in a location in the {\tt \$PATH} it is enough to
run\footnote{Despite the name, this has nothing to do with {\tt autotools}.}
\begin{lstlisting}
  ./configure
\end{lstlisting}
For non-standard installation the paths can be specified like this
\begin{lstlisting}
  ./configure --with=hoppet=/path/to/hoppet-config --with-lhapdf=/path/to/lhapdf-config
\end{lstlisting}
In both cases the file {\tt Makefile.inc} is created which can also be
modified by hand. The program should now compile with
\begin{lstlisting}
  make [-j]
\end{lstlisting}
which will result in the binary {\tt disorder} being created.
\subsubsection{The analysis framework}
The code uses the {\tt POWHEG-BOX} analysis framework, with a few
minor modifications. A few example analyses are included in the {\tt
  analysis directory}. Any new analysis should be put here and name of
the analysis should to be specified in the {\tt Makefile}. For
example, if the file {\tt my\_analysis.f} is to be compiled one should
add
\begin{lstlisting}
  ANALYSIS += my_analysis
\end{lstlisting}
to the {\tt Makefile} in place of the default {\tt
  simple\_analysis}. There are two mandatory routines in the analysis
file, {\tt define\_histograms} and {\tt user\_analysis}. In the first
routine one should define histograms like this (there are also
routines available to book histograms with varying bins sizes)
\begin{lstlisting}
  call bookupeqbins('string_name', binsize, min, max)
\end{lstlisting}
In the {\tt user\_analysis} routine the user should perform their
analysis and fill histograms like this
\begin{lstlisting}
  call filld('string_name', obs_value, dsig)
\end{lstlisting}
where {\tt obs\_value} is the value of the observable to be binned, and
{\tt dsig} is the associated array of weights. The routine takes as
input
\begin{lstlisting}
  integer n
  double precision dsig(maxscales), x, y, Q2
\end{lstlisting}
where {\tt n} is the number of initial plus final state particles,
{\tt dsig} is the weights computed by \disorder{} and {\tt maxscales}
is the maximum number of scales which is supported (currently 7). {\tt
  x}, {\tt y}, and {\tt Q2} are the DIS variables.

Through the module {\tt mod\_analysis} the analysis has access to two
array of momenta {\tt pbreit(0:3,1:n)} and {\tt plab{0:3,1:n}} which
are the Breit and lab frame momenta respectively. The output of the
analysis will be saved to the disk as outlined below.

\subsection{Inclusive mode}
\label{sec:Inclusive}
The syntax for running the program is
\begin{lstlisting}
  ./disorder [options]
\end{lstlisting}
Running the program without any options will compute the total
inclusive cross section using the default parameters of the
program, which are printed to screen. The user can get a list of most
parameters which can be specified on the command line by running
\begin{lstlisting}
  ./disorder -help
\end{lstlisting}
For a complete list of all parameters, and their use, the user should
look through the file {\tt src/mod\_parameters.f90}. Here we describe
the most common flags.

The program allows the user to specify limits on $x$, $y$, and $Q$
through {\tt -xmin}, {\tt -xmax}, {\tt -ymin}, {\tt -ymax}, {\tt
  -Qmin}, {\tt -Qmax}, or to fix them through the options {\tt -x},
{\tt -y}, {\tt -Q}. For instance to compute the cross section at $Q=20
\, \mathrm{GeV}$ and $x>0.01$ one would run
\begin{lstlisting}
  ./disorder -Q 20 -xmin 0.01
\end{lstlisting}
The program will perform a Monte Carlo integration in the ranges
specified, using the integrator VEGAS~\cite{Lepage:1977sw}. If the
phase space is fully constrained by fixing two of either $x$, $y$, and
$Q$ the program simply evaluates one point and returns the answer.

One can further specify the energy of the incoming lepton through {\tt
  -Elep}, the incoming hadron through {\tt -Ehad}. By default the
lepton is taken to be an electron but specifying {\tt -positron} on
the command line will change that. By default the code computes the
photon-mediated NC cross section. To include the $Z$ one can specify
{\tt -includeZ} and to include CC processes one can specify {\tt
  -CC}. The inclusion of NC processes can also be controlled through
the {\tt -NC} flag. In fact, all logical flags can be prefixed by
``no'' to turn them off. Hence the below command line would run the
program with CC processes only using a positron
\begin{lstlisting}
  ./disorder -Q 20 -xmin 0.01 -noNC -CC -positron
\end{lstlisting}

The order of the calculation is by default NNLO but can be specified
with one of the flags {\tt -lo/-nlo/-nnlo/-n3lo}. The PDF, using the
LHAPDF name, can be specified with the {\tt -pdf} flag. If the user
wants to compute PDF uncertainties, the flag {\tt pdfuncert} should be
given. This flag will make \disorder{} loop over all the members in
the PDF, and combine their errors according to the internal LHAPDF
routine {\tt getpdfuncertainty}. This also means that the program is
slowed down proportionally to the number of PDF members. If the PDF
also includes $\as$ variations these are included in the PDF
uncertainty by default. If the user wants the PDF and $\as$
uncertainties independently, then the flag {\tt -alphasuncert} should
be specified. Some care should be taken here, as in practice the code
assumes that the $\as$ variations are contained in the last two PDF
members, and simply separates them from the rest.

Renormalisation, $\mu_R$ and factorisation, $\mu_F$ scale
uncertainties can be included by specifying the flag {\tt
  -scaleuncert}. \disorder{} uses the vector boson virtuality, $Q$, as
its default central scale. The program will then compute a standard
7-point scale variation varying this scale by a factor of two up and
down keeping $1/2\le \mu_R/\mu_F\le 2$. On the screen the envelope of
all 7 runs will be printed. The user can in principle also carry out
arbitrary variations in individual runs by specifying {\tt -xmur} and
{\tt -xmuf} on the command line. Here {\tt xmur} is the ratio of
$\mu_R/Q$ and similarly for {\tt xmuf}.\footnote{In practice it is
faster to use the {\tt scaleuncert} flag as the program will only
recompute what is needed for the variations rather than do a full
event. However, at N3LO, in general, this is not true, due to the fact
that the number of tables needed in \hoppet{} to carry out on-the-fly
scale variations increases dramatically at this order, compared to
using a fixed ratio of $Q$. Given how fast the code is, it is still
the author's opinion that it is more convenient to use the on-the-fly
variations.}

Finally the random seed can be set with the {\tt iseed} flag. When the
program terminates it will print results to screen but also save a
number of files depending on the exact input. There is always a file
called {\tt xsct\_nnlo\_seed0001.dat}, where the {\tt nnlo} and {\tt
  seed0001} parts will vary depending on the order and seed, which
contains a summary of the run, including the total and reduced cross
sections.

The output of the analysis is printed to {\tt .dat} files prefixed by
{\tt disorder\_nnlo\_seed001\_pdfmem000} where again the exact prefix
will depend on the input. The output name also contains information on
the seed, the PDF member and the renormalisation and factorisation
scales if {\tt -scaleuncert} is on. The output prefix can be
overwritten by specifying a string name with the flag {\tt -out}. Care
should be taken as this will not work if running with {\tt -pdfuncert}
on.

\subsection{P2B mode}
\label{sec:P2Bmode}
To turn on P2B it is enough to specify {\tt -p2b} on the command
line. Many of the flags described above can also be specified in this
mode, with a few limitations
\begin{itemize}
\item {\tt -CC} and {\tt -includeZ} are not supported
\item {\tt -n3lo} is not supported
\item {\tt -pdfuncert} and {\tt -alphasuncert} are not supported
\item {\tt -y}, {\tt -ymin}, and {\tt -ymax} are not supported
\end{itemize}
When running in P2B mode it is furthermore of use to be able to
control the number of calls to \disent{}. This is done through the
flag {\tt -ncall2} which takes an integer. It can in principle also
be specified in the inclusive mode, but most users should not need
too.

We provide a small script to run on multiple cores on a single machine
in {\tt aux/runpar.sh}. In the {\tt aux} folder one may also find a
script called {\tt mergedata} that can perform various manipulations
of the datafiles. In particular
\begin{lstlisting}
  ./mergedata 1 {list of statistically equivalent files}
\end{lstlisting}
will take the average of all files and produce the file {\tt fort.12}
with the result. Running the script without any arguments will result
in a list of possible uses of the script. The {\tt mergedata} script
is taken from the {\tt POWHEG-BOX}. 

\section{Benchmarks and results}
\label{sec:results}

\section{Conclusion}
\label{sec:conclusion}

\section*{Acknowledgements}
The author is grateful for discussions with Gavin Salam regarding {\tt disent}. 


\begin{appendix}

  \section{An appendix}

\end{appendix}

\bibliography{bibliography.bib}

\nolinenumbers

\end{document}
