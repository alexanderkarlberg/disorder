% =========================================================================
% SciPost LaTeX template
% Version 2021-08
%
% Submissions to SciPost Journals should make use of this template.
%
% INSTRUCTIONS: simply look for the `TODO:' tokens and adapt your file.
%
% You can also make use of our empty "skeleton" templates for each Journal,
% e.g. SciPostPhys_skeleton.tex
% =========================================================================
\documentclass[submission, PhysCodeb]{SciPost}

% Prevent all line breaks in inline equations.
\binoppenalty=10000
\relpenalty=10000

\hypersetup{
    colorlinks,
    linkcolor={red!50!black},
    citecolor={blue!50!black},
    urlcolor={blue!80!black}
}

\usepackage[bitstream-charter]{mathdesign}
\urlstyle{sf}

% Fix \cal and \mathcal characters look (so it's not the same as \mathscr)
\DeclareSymbolFont{usualmathcal}{OMS}{cmsy}{m}{n}
\DeclareSymbolFontAlphabet{\mathcal}{usualmathcal}

\newcommand{\hoppet}{{\sc hoppet}}
\newcommand{\disent}{{\tt disent}}
\newcommand{\disorder}{{\tt disorder}}
\newcommand{\provbfh}{{\tt proVBFH}}
\newcommand{\disaster}{{\tt disaster++}}
\newcommand{\nlojet}{{\tt nlojet++}}
\newcommand{\as}{\alpha_{\mathrm{s}}}
\newcommand{\dd}{\mathrm{d}}
\newcommand{\NC}{\mathrm{NC}}
\newcommand{\CC}{\mathrm{CC}}

\begin{document}
\begin{flushright}
CERN-TH-2023-???
\end{flushright}
\begin{center}{\Large \textbf{
disorder: Deep inelastic scattering at high orders\\
}}\end{center}

\begin{center}
Alexander Karlberg
\end{center}

\begin{center}
CERN, Theoretical Physics Department, CH-1211 Geneva 23, Switzerland
\\
{\small \sf alexander.karlberg@cern.ch}
\end{center}

\begin{center}
\today
\end{center}

% For convenience during refereeing (optional),
% you can turn on line numbers by uncommenting the next line:
%\linenumbers
% You should run LaTeX twice in order for the line numbers to appear.

\section*{Abstract}
{\bf
% TODO: write your abstract here.
We present a Fortran 77/95 code capable of computing QCD corrections
in deep inelastic scattering (DIS). The code uses the
Projection-to-Born method to combine an exclusive DIS + 1 jet
calculation with the inclusive DIS structure functions, thereby
obtaining fully differential DIS predictions at
$\mathcal{O}(\as^2)$. The code is lightweight and fast, and yet
includes the most common functionalities found in typical perturbative
QCD programs, like automatic renormalisation and factorisation scale
uncertainties, options to run and combine multiple seeds, and
interfaces to {\tt fastjet} and {\tt LHAPDF}. Due to the underlying
exclusive DIS + 1 jet code, the program also provides stable results
in the infrared, relevant for extracting logarithmic coefficients for
analytic resummations. As a by-product the code provides access
to the DIS structure functions and (reduced) cross sections up to
$\mathcal{O}(\as^3)$.}


% TODO: include a table of contents (optional)
% Guideline: if your paper is longer that 6 pages, include a TOC
% To remove the TOC, simply cut the following block
\vspace{10pt}
\noindent\rule{\textwidth}{1pt}
\tableofcontents\thispagestyle{fancy}
\noindent\rule{\textwidth}{1pt}
\vspace{10pt}

\section{Introduction}
\label{sec:intro}
Deep inelastic scattering (DIS) is arguably one of the best understood
processes in perturbative QCD. It is one of few processes for which
there exists an exact factorisation
theorem~\cite{Collins:1987pm,Collins:1989gx}, and the massless
unpolarised hard perturbative coefficients have been computed through
an impressive three
loops~\cite{SanchezGuillen:1990iq,vanNeerven:1991nn,Zijlstra:1992qd,Zijlstra:1992kj,vanNeerven:1999ca,vanNeerven:2000uj,Moch:1999eb,Moch:2004xu,Vermaseren:2005qc,Vogt:2006bt,Moch:2007rq,Davies:2016ruz}
with progress on the four loop results presented recently in
Ref.~\cite{Moch:2022frw}. Together with the three-loop results for the
splitting functions~\cite{Moch:2004pa,Vogt:2004mw} and the four-loop
$\beta$-function~\cite{vanRitbergen:1997va,Czakon:2004bu} for the
running coupling this allows for the determination of the proton
structure functions at the next-to-next-to-next-to-leading order
(N3LO).\footnote{Technically the four-loop splitting functions are
needed to claim this accuracy. Recent progress in determining those
can be found in
Refs.\cite{Moch:2021qrk,Falcioni:2023luc,Falcioni:2023vqq,Gehrmann:2023cqm}.}
The structure functions can be combined with an exclusive
next-to-next-to-leading order (NNLO) DIS + 1 jet computation to obtain
fully differential N3LO predictions in DIS, as was done by the NNLOJET
collaboration~\cite{Currie:2018fgr,Gehrmann:2018odt}. This computation
uses the Projection-to-Born (P2B) method, which was first introduced
in the context of NNLO Vector Boson Fusion (VBF)
production~\cite{Cacciari:2015jma}.

Despite this impressive theoretical progress, there are few publically
available computer codes from which one can obtain
fast and reliable high-order cross section predictions. This paper
and associated Fortran code seeks to address that.

One of the advantages of the P2B method is that it is rather agnostic
towards the details of the underlying exclusive computation, and it is
hence possible to ``upgrade'' existing fixed-order codes to one order
higher.

Historically the most succesful fixed-order codes have been
\disent{}~\cite{Catani:1996vz}, \disaster{}~\cite{Graudenz:1997gv},
and \nlojet{}~\cite{Nagy:2001xb} which are all next-to-leading order
(NLO) accurate. Discrepancies between \disent{} and fixed-order
coefficients from analytical resummation were initially observed in
Refs.\cite{Antonelli:1999kx,Dasgupta:2002dc}, and were only recently
understood to be due to a bug in one of the dipole terms for the gluon
channel~\cite{Borsa:2020ulb,Borsa:2020yxh}.

With the bug fixed we can use \disent{} as the underlying exclusive
NLO code together with the NNLO structure functions
from~\hoppet{}~\cite{Salam:2008qg,BertoneKarlberg}.\footnote{The
combination of \disent{} and \hoppet{} naturally leads to a Fortran
code. For a C++ alternative, based on publically available codes, one
could have started from either \disaster{} or \nlojet{} and used the
structure functions as implemented in
APFEL++~\cite{Bertone:2013vaa,Bertone:2017gds}.} Advantages of using
\disent{} are its well-known efficiency and numerical stability. The
resulting program is dubbed \disorder{} keeping in the spirit of
previous names for fixed-order DIS codes.\footnote{{\tt
  dispatch}~\cite{Dasgupta:2002dc} deserves an honorable mentioning in
this context.} Since the program relies on \hoppet{} for the structure
functions, the inclusive DIS cross section can actually be obtained
one order higher, since the structure functions were implemented at
N3LO already in the context of the {\tt proVBFH}
code~\cite{Cacciari:2015jma,Dreyer:2016oyx,Dreyer:2018qbw,Dreyer:2018rfu}. The
\disorder{} program therefore has two more or less separate use cases:
1. The computation of fully exclusive NNLO photon-mediated DIS 2. The
computation of inclusive/reduced neutral- (NC) and charged-current
(CC) DIS at N3LO accuracy.

The paper is structured as follows: In section~\ref{sec:dis} we review
the DIS process and kinematics as implemented in \disorder{}. In
section~\ref{sec:running} we provide details on how to run \disorder{}
and in section~\ref{sec:results} we show a few results from the
program. Finally we conclude in section~\ref{sec:conclusion}.



\section{Basics of the DIS process}
\label{sec:dis}
At leading order (LO) the DIS process is the scattering of a massless
(anti-)quark $q$ off a massless \mbox{(anti-)lepton} $l$ via the
exchange of a photon or electroweak gauge boson $V$ of virtuality
$Q^2$. Denoting the external four-momenta by $k_i$ (incoming lepton),
$k_f$ (outgoing lepton), $p_i$ (incoming quark), and $p_f$ (outgoing
quark) we can define the lorentz invariant DIS variables $x$, $Q^2$,
and $y$, given by
\begin{align}
  Q^2 = -q^2 = -(k_i - k_f)^2, \qquad  x = \frac{Q^2}{2 P \cdot q},
  \qquad  y = \frac{P \cdot q}{P \cdot k_i} = \frac{p_i \cdot q}{p_i \cdot k_i} ,
  \label{eq:dis-variables}
\end{align}
where $P$ is the proton four-momentum. These variables fully specify
the kinematics at LO. DIS is most often analysed in the Breit-frame
which is specified\footnote{For the explicit transformation between
lab- and Breit-frames we follow Appendix 7.11 in
Ref.~\cite{Devenish:2004pb}.} by requiring that $2 x \vec{P} + \vec{q}
= 0$. In this frame the mediated vector boson has zero energy
component and is anti-aligned with the incoming parton. Explicitly in
\disorder{} the Breit-frame at LO is
\begin{align}
k_i &= \frac{Q}{2}\left(\frac{2-y}{y},\frac{2\sqrt{1-y}}{y},0,-1 \right), \quad  p_i = \frac{Q}{2}\left(1,0,0,+1 \right) \\ \notag
k_f &= \frac{Q}{2}\left(\frac{2-y}{y},\frac{2\sqrt{1-y}}{y},0,+1 \right), \quad  p_f = \frac{Q}{2}\left(1,0,0,-1 \right) 
\end{align}
where four-momenta are given as $(E,p_x,p_y,p_z)$. The resulting
vector $q$ is hence given by $Q(0,0,0,-1)$. In the lab frame we align
the parton with the positive $z$-axis and hence the lepton with the
negative
\begin{align}
\tilde{k}_i &= E_\mathrm{l}\left(1,0,0,-1 \right), \quad  \tilde{p}_i = x E_\mathrm{h}\left(1,0,0,+1 \right) 
\end{align}
where $E_\mathrm{h}$ is the energy of the incoming hadron. Using the
definitions of Eq.~\eqref{eq:dis-variables} one finds that in the lab
frame
\begin{equation}
  \tilde{q} = \left(y(E_\mathrm{l} - x E_\mathrm{h}), -Q\sqrt{1-y},0, -y(E_\mathrm{l} - x E_\mathrm{h})\right) 
\end{equation}
and the outgoing momenta then simply follow from momentum conservation
\begin{equation}
  \tilde{k}_f = \tilde{k}_i - \tilde{q}, \quad   \tilde{p}_f = \tilde{p}_i + \tilde{q}\,. 
\end{equation}

The inclusive cross section for DIS can be split into a NC
contribution, from $e^\pm p \to e^\pm + X $ scattering, and a CC
contribution from $e^\pm p \to \nu + X$\footnote{Here
$\nu=\{\nu_e,\bar{\nu}_e\}$ for $e^-$ and $e^+$ respectively. One can
of course also consider incoming neutrinos which does not change the
discussion here, but has not been implemented in \disorder{}.}
scattering. The unpolarised NC cross section can be written as
\begin{equation}
\frac{\dd\sigma_{\NC}^\pm}{\dd x \dd Q^2} =   \frac{2\pi\alpha^2}{xQ^4} \left[y_+ F_2^{\NC} \mp y_- F_3^\NC - y^2 F_L^\NC\right],\,
\label{eq:NCsigma}
\end{equation}
where $y_\pm=1\pm(1-y)^2$, $\alpha$ is the fine structure constant and
$F_i^\NC$ can be expressed in terms of the usual proton structure
functions
\begin{align}
  F_{i}^\NC &= F_{i}^\gamma  - v_e \Gamma_{\gamma Z} F_{i}^{\gamma Z} + (v_e^2 + a_e^2)\Gamma_{Z} F_{i}^{Z},\, \quad i =2,L \\ \notag
  F_{3}^\NC &= - a_e \Gamma_{\gamma Z} F_{3}^{\gamma Z} + 2v_e a_e\Gamma_{Z} F_{3}^{Z}\,.
\end{align}
Here $v_e = -\frac12 + 2 \sin^2\theta_W$, $a_e=\frac12$, $\Gamma_{\gamma Z} = \frac{Q^2}{\sin^2 2\theta_W(Q^2+M_Z^2)}$, and $\Gamma_Z=\Gamma_{\gamma Z}^2$


\section{Running \disorder{}}
\label{sec:running}

\section{Benchmarks and results}
\label{sec:results}

\section{Conclusion}
\label{sec:conclusion}

\section*{Acknowledgements}
The author is grateful for discussions with Gavin Salam regarding {\tt disent}. 


\begin{appendix}

  \section{An appendix}

\end{appendix}

\bibliography{bibliography.bib}

\nolinenumbers

\end{document}
